\begin{acronym}
        \acro{circuitvoyager}[CircuitVoyager]{The Name of the DMM I'm developing.}
        \acro{devboard}[DevBoard]{main microcontroller developement board. (STM32H747I-Disco)}
        \acro{dmm}[DMM]{digital multimeter}
        \acro{hw}[HW]{Hardware}
        \acro{sw}[SW]{Software}
        \acro{spi}[SPI]{Serial Peripheral Interface (low level protocol)}
        \acro{sdram}[SDRAM]{Synchronous Dynamic Random Access Memory (external RAM)}
        \acro{ui}[UI]{User Interface}
        \acro{mcu}[MCU]{Micro Controlling Unit}
        \acro{mipidsi}[Mipi DSI]{Digital Serial Interface (Display Protocol)}
        \acro{fat}[FAT]{File Allocation System (Low Level Filesystem)}
        \acro{hal}[HAL]{Hardware Abstraction Layer (STM32 Abstraction Library)}
        \acro{ethz}[ETHZ]{Eidgenössische Technische Hochschule}
        \acro{tbz}[TBZ]{Technische Berufsschule Zürich}
        \acro{adc}[ADC]{Analog Digital Converter}
        \acro{tim}[TIM]{Timer (Hardware Block in STM32)}
        \acro{pcb}[PCB]{Printed Circuit Board}
        \acro{dut}[DUT]{Device under test}
        \acro{ovp}[OVP]{Over voltage protection}
        \acro{touchgfx}[TouchGFX]{Graphical \acs{ui} designer for STM32 \acs{mcu}s}
        \acro{qspi}[QPSI]{Quad \acs{spi}}
        \acro{ol}[OL]{Overload}
        \acro{mux}[MUX]{Multiplexer}
\end{acronym}