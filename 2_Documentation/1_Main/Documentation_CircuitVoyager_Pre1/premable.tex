% ============= Packages =============

% Switch between German and English such as the title page design based on the main.tex. file settings
\usepackage{ifthen}


\usepackage[english]{babel} % English


% Standard Packages
\usepackage[utf8]{inputenc}
\usepackage{graphicx} % Used to insert images
\usepackage{float} %Package for using the [H] option on graphics to force them into place
\usepackage[export]{adjustbox} % Used to constrain images to a maximum size
\usepackage{enumerate} % Needed for markdown enumerations to work
\usepackage{geometry} % Used to adjust the document margins
\usepackage{amsmath} % Equations
\usepackage{amssymb} % Equations
\usepackage[mathletters]{ucs} % Extended unicode (utf-8) support
\usepackage{fancyvrb} % verbatim replacement that allows latex
\usepackage{grffile} % extends the file name processing of package graphics
\usepackage{url}
\def\UrlBreaks{\do\/\do-}
\usepackage{hyperref}
\usepackage{longtable} % longtable support required by pandoc >1.10
\usepackage{multicol}
\usepackage{multirow}
\usepackage[nohyperlinks]{acronym}
\usepackage{enumitem}
\usepackage{fixmath}
\usepackage{blindtext}
\usepackage[T1]{fontenc}
\usepackage{graphicx}
\usepackage{subfigure}
\graphicspath{{Figures/}} % define path of images
\usepackage{animate}
\usepackage{fancyhdr}
\usepackage{lmodern}
\usepackage[dvipsnames]{xcolor}
\usepackage{footnote}
\usepackage{lastpage}

\usepackage{pgfgantt}

% Citation style
\bibliographystyle{ieeetr}

% BlockDiagram Drawing Package
\usepackage{tikz}
\usetikzlibrary{shapes,arrows}
\usepackage[european,nooldvoltagedirection]{circuitikz}
\usepackage{pgfplots}
\pgfplotsset{compat=1.10}
\usepackage{mathtools}
\usepackage{pgf-pie}
\usepackage{textcomp}
\usepackage{pgfplots}
\usepgfplotslibrary{statistics}
\usepackage{smartdiagram}

%  ============= BlockDiagram Drawing Config =============
% Definition of blocks:
\tikzset{%
  block/.style    = {draw, thick, rectangle, minimum height = 3em,
    minimum width = 3em},
  sum/.style      = {draw, circle, node distance = 2cm}, % Adder
  input/.style    = {coordinate}, % Input
  output/.style   = {coordinate}, % Output
  mult/.style	  = {draw, isosceles triangle, minimum height=1cm, minimum width =1cm}
}

%  ============= Settings for listings  =============
\usepackage{listings,lstautogobble}
\lstset{title=\lstname, frame=single, framerule=0pt, rulecolor=\color{lightgray}, showspaces=false, showstringspaces=false, showtabs=false, numbers=left, numbersep=5pt, numberstyle=\sffamily\tiny\color{gray}, breaklines=false, autogobble=true, basicstyle=\sffamily\scriptsize}

\usepackage[breakable]{tcolorbox}
\newtcolorbox{codeblock}{
    colback=gray!5!white,
    colframe=gray!95!black,
    before skip=20pt,
    after skip=20pt
    }
\newtcolorbox{codeblock-b}{
    breakable,
    colback=gray!5!white,
    colframe=gray!95!black,
    before skip=20pt,
    after skip=20pt
    }
\newtcolorbox{box-black}{
    colback=black,
    colframe=black,
    before skip=10pt,
    after skip=10pt
    }

%  ============= Color definition ============= 
% FH-Blau
\definecolor{FH}{RGB}{8, 64, 126}
\definecolor{FH2}{RGB}{8, 64, 126}

% ============= No massive space betweend headline and chapter titel =============
\renewcommand*{\chapterheadstartvskip}{\vspace*{-.4cm}}
\renewcommand*{\chapterheadendvskip}{\vspace{.5cm}}

\setlength{\parindent}{0pt} % no indent at paragraph start

% ============= Header and Footer =============
\renewcommand{\chaptermark}[1]{\markboth{\thechapter~ #1}{}}

\fancypagestyle{icmt-fancy}{%
  \fancyhf{}% Clear header and footer
  %\fancyhead[L]{\leftmark}
  \fancyhead[L]{\subTitle}
  \fancyhead[R]{\dateDay.\dateMonth.\dateYear}
  \fancyfoot[L]{\studentFirstNameone~\studentLastNameone}
  \fancyfoot[C]{TBZ / ETHZ}
  \fancyfoot[R]{\thepage}% Custom footer *~|~\pageref{LastPage}
  \renewcommand{\headrulewidth}{0.4pt}% Line at the header visible
  \renewcommand{\footrulewidth}{0.4pt}% Line at the footer visible
}

% Define horizontal lines above and below the main content
\makeatletter
\def\thickhline{%
	\noalign{\ifnum0=`}\fi\hrule \@height \thickarrayrulewidth \futurelet
	\reserved@a\@xthickhline}
\def\@xthickhline{\ifx\reserved@a\thickhline
	\vskip\doublerulesep
	\vskip-\thickarrayrulewidth
	\fi
	\ifnum0=`{\fi}}
\makeatother
\newlength{\thickarrayrulewidth}
\setlength{\thickarrayrulewidth}{2\arrayrulewidth}

% ============= Redefine the plain page style =============
\fancypagestyle{plain}{%
  \fancyhf{}%
  \fancyfoot[R]{\thepage}%
  \renewcommand{\headrulewidth}{0.0pt}% Line at the header invisible
  \renewcommand{\footrulewidth}{0.0pt}% Line at the footer visible
}
\renewcommand*{\chapterpagestyle}{icmt-fancy}

% ============= Size of headings =============
\setkomafont{chapter}{\LARGE}
\setkomafont{section}{\Large}
\setkomafont{subsection}{\large}
\setkomafont{subsubsection}{\normalsize}
\setkomafont{paragraph}{\normalsize}
\setkomafont{subparagraph}{\small}

% roman numbering with \RM{Zahl}
\newcommand{\RM}[1]{\MakeUppercase{\romannumeral #1}}

% ============= hyperref should always be added at the end =============
\usepackage{hyperref}

\hypersetup{
    unicode=false, % non-Latin characters in Acrobat’s bookmarks
    pdftoolbar=true, % show Acrobat’s toolbar?
    pdfmenubar=true, % show Acrobat’s menu?
    pdffitwindow=true, % window fit to page when opened
    pdfstartview={FitV}, % fits the width of the page to the window
    pdfpagelayout={SinglePage}, % displays only one page when opened in pdf viewer
    pdftitle={\workTitle},
	pdfsubject={\subTitle},
	pdfauthor={\studentFirstNameone~\studentLastNameone},
	pdfkeywords={\workTitle,~\subTitle},
    pdfcreator={pdflatex}, % creator of the document
    pdfproducer={LaTeX with hyperref}, % producer of the document
    pdfnewwindow=true, % links in new window
    colorlinks=false, % false: boxed links; true: colored links
    linkcolor=black, % color of internal links (change box color with linkbordercolor)
    citecolor=black, % color of links to bibliography
    filecolor=black, % color of file links
 	breaklinks=true,	
	menucolor=black,
    urlcolor=black % color of external links
}
