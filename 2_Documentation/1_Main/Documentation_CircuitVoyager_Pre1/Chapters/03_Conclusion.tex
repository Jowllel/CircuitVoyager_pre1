\chapter{Conclusion}
\label{cha:Conclusion}


\textbf{Project Start}

I've noticed that it was quite hard for me to start with the project. There are so much topics to imply, that it can get overwhelming and very emotional. I've also underestimated the time needed to set up the documentation LaTeX files and the whole planning. It can be demotivating if you're spending a lot of time and in the end didn't really start with the project. But later I was able to start the project without much trouble. 

\vspace{5mm}
\textbf{Time Plan}

Well, the GANTT chart as you can see didn't help out much for me. Because it's really hard to plan such a projects timing over half a year. There are so many unsolved problems and in the beginning I haven't been able to estimate the duration of all the processes involved. On the other hand my 2-week plans were working great and helped a lot, to notice what was to do.

\vspace{5mm}
\textbf{HW development}

I've learned much in this project part. Mainly this was Altium Designer, as this was my first complete project I've realized in Altium Designer. This also leaded to some not nicely solved solution. All components for example have their own properties and this leads to a unreadable BOM.

But in the end I've also noticed, that the dataflow (that usually should go from left to right) goes in the wrong direction. This had already started in the HW-Chart and therefore also ended up in the schematic.

I've also used Draw.io one of the first times and by now it looks very promising. It's much easier and more straight forward than MS Visio. Everything just works as intended.

\vspace{5mm}
\textbf{SW development}

Because in this project I've used a dual-core MCU for the first time, it was really hard to write the code at first. And with time I noticed that it's to complicated for me to implement such modern techniques. This is also due to the bad documentation of tools like FreeRTOS TouchGFX or multicore MCUs.

\newpage
\textbf{Project Idea}

The Project Idea in the first was really good. But as you can see, some minor decisions weren't as intelligent as they could be. I noticed that I'm much better at the "old style" electronics, without any operating system, overpowered MCUs, or TouchGFX. But I'm happy, that I was able to switch the focus of my project from hi-speed protocols to more embedded software and LaTeX. In the end I have a product that I'm proud of and that works. But I won't move on with the Circuit Voyager in the next semester, because I also noticed, that It's rather hard to make the hardware development of measurement equipment. This takes a lot of time, also away from school. And in the next semester I want to focus more on stuff like Swiss Skills and therefore make an easier project.

\vspace{5mm}
\textbf{Outlook}

As described in the project idea chapter. I want to stop the development of the Circuit Voyager and move on with an easier more embedded and hardware close project with simpler protocols and so on. But on the other hand I'll also document my further projects with LaTeX, because I got a lot faster, and it's more predictable than MS Word.